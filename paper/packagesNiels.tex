

\usepackage[utf8]{inputenc} %Windows, Linux
%\usepackage[ansinew]{inputenc} %TeXnicCenter kann leider kein UTF-8

%Umlaute
%\usepackage[english]{babel} % can be deleted? british?

%Zitationsstiel
\usepackage[authoryear]{natbib}





% Standardschriftart
%\usepackage{times}
%\usepackage{garamond}

% Default-Schriften (\rmdefault Times, 
%\sfdefault Helvetica und \ttdefault Courier
%besser als \usepackage{times} da auch Mathematikschriften 
%berücksichtigt werden
%\usepackage{mathptmx}  %Times CAREFUL: CHANGES \mathcal{M}!!
\usepackage[scaled=.92]{helvet}
%\usepackage{mathpazo} %Palatino
%\usepackage[scaled=.96]{berasans}
\usepackage{courier}

%Grafiken
\usepackage{graphicx} % z.B. \resizebox
\usepackage{subfigure} 
% Tabellen
\usepackage{booktabs}
\usepackage[table]{xcolor} % Zellen in Tabellen farbig markieren

% Grafiken in einem Unterordner speichern
\graphicspath{{./images/}}





% Seitenränder
\usepackage[a4paper]{geometry}
\geometry{top=26mm, bottom=19mm, left=40mm, right=35mm, includefoot}

% Seitenränder auf Deckblatt
\usepackage[strict]{changepage}

%Querseiten mit korrekter Kopf-/Fußzeile
\usepackage{pdflscape}


%sauberer Blocksatz, optischer Randausgleich
\usepackage{microtype}


%1,5facher Zeilenabstand nur im Fließtext
\usepackage{setspace}
\onehalfspacing


% Mathematisches
%\usepackage{amssymb}
\usepackage{amsmath} 
%\usepackage{amsfonts} 
\allowdisplaybreaks %page break in equation

% Code und Pseudocode
%\usepackage[ruled]{algorithm}
%\usepackage{algpseudocode}
%\usepackage[boxed]{algorithm}
%\usepackage{algpseudocode}

%Silbentrennung
\usepackage[T1]{fontenc}



\usepackage{url}

\usepackage{color}
% Wofür wichtig? toc, lof, lot?
%\usepackage{colortbl} %Für Listings
%  \definecolor{dunkelgrau}{rgb}{0.7,0.7,0.7}
%  \definecolor{hellgrau}{rgb}{0.9,0.9,0.9}
%
%\usepackage{listing}
%  \renewcommand{\listlistingname}{Quelltextverzeichnis}
%
%\usepackage{listings}
%  \lstset{numbers=left, numberstyle=\tiny, basicstyle=\scriptsize, backgroundcolor=\color{hellgrau}}
%  \lstset{captionpos=b, aboveskip=15pt, belowskip=8pt, showstringspaces=false}



%Abkürzungsverzeichnis, nur verwendete Abkürzungen darstellen
\usepackage[printonlyused]{acronym}

%Anhang
\usepackage{appendix}


%Bild-, Tabellenuntschriften schöner formatieren
\usepackage[format=hang,margin=10pt,font=small,labelfont=bf]{caption}

%PDF einbinden
%\usepackage{pdfpages}
%\includepdf[pages=1-4]{Meindoku.pdf}


%Euro 
%\usepackage{eurosym} 
% das Euro-Zeichen kann so eingefügt werden: \euro{}

%Lorem ipsum Auto-Text um Format zu zeigen
\usepackage{blindtext}


% Abstand zwischen Absätzen
%\parskip 3pt

% kein Erstzeileneinzug
%\setlength{\parindent}{0em}

% Tabellen
\usepackage{tabularx}				% zusätzliche Optionen in \begin{tabular} ... \end{tabular}
	\newcolumntype{L}[1]{>{\raggedright\arraybackslash}p{#1}} 	% linksbündig mit Breitenangabe z.B. L{3cm}
	\newcolumntype{C}[1]{>{\centering\arraybackslash}p{#1}} 	% zentriert mit Breitenangabe z.B. C{3cm}
	\newcolumntype{R}[1]{>{\raggedleft\arraybackslash}p{#1}} 	% rechtsbündig mit Breitenangabe z.B. R{3cm}

